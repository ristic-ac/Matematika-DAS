\begin{abstract}
\doublespacing
У раду се испитује дестилација модела машинског учења у скуп логичких правила помоћу ИЛП система \emph{Aleph}. Коришћени су скупови \emph{Mushroom} и \emph{Adult} са \emph{OpenML}-а. Претпроцесирање обухвата дискретизацију нумеричких атрибута и балансирање класа \emph{undersampling}-ом. Као учитељи су примењени \emph{Decision Tree}, \emph{Random Forest} и \emph{XGBoost}. Избор хиперпараметара је извршен \emph{grid search} поступком са 5-струком унакрсном валидацијом и критеријумом \emph{accuracy}. Дестилација је спроведена у три пресета (\textit{sniper}, \textit{sweet\_spot}, \textit{sweeper}). Оцењене су метрике \emph{accuracy}, \emph{precision}, \emph{recall}, \emph{F1}, \emph{MCC} и фиделитет према учитељу. На скупу \emph{Mushroom} постижу се веома високи резултати уз мали број правила. На скупу \emph{Adult} фиделитет често премашује тачност према истини, уз умерене добити од повећања сложености. Ансамбл учитељи (\emph{RF}, \emph{XGB}) у просеку дају стабилније дестилате од \emph{DT}-а.\end{abstract}
\pagebreak