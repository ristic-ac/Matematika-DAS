\section{Закључак}
У раду је приказан поступак дестилације учитељских модела у логичка правила помоћу \emph{Aleph}-а, са систематском анализом метрика квалитета и показатеља сложености на скуповима \emph{Mushroom} и \emph{Adult}. На скупу \emph{Mushroom} добијени су готово максимални резултати уз мали број кратких правила, што указује да се висока тачност и фиделитет могу остварити без повећања сложености и да Парето фронт рано достиже „колено“. На скупу \emph{Adult} уочава се стабилнија појава фиделитета изнад стварне тачности, што указује да дестилати веома добро имитирају понашање учитеља, али да додатна сложеност после одређене тачке доноси само ограничене добити у односу на реалну генерализацију. Ансамбл учитељи (\emph{RF}, \emph{XGB}) у просеку доводе до дестилата са стабилнијим метрикама у односу на \emph{DT}, док пресет \textit{sweet\_spot} често обезбеђује повољан однос између квалитета и интерпретабилности при умереним ограничењима дужине клаузе и строжем прагу \texttt{minacc}. Практична препорука је примена умерено рестриктивних поставки и провера осетљивих метрика у условима неуједначених класа. Неопходно је и извршити проверу стабилности на варијацијама дискретизације и семена. Као потенцијални даљи рад се предвиђа испитивање алтернативног редоследа претпроцесирања и механизама контроле компромиса прецизност-одзив на нивоу ансамблирања.